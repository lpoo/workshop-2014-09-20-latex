\documentclass{article}
\usepackage[utf8]{inputenc}
\begin{document}
% Produza a seguinte tabela
%
% +--------+--------------------------+---------+---------+-------+
% | RA     | Aluno                    | Nota P1 | Nota P2 | Média |
% +--------+--------------------------+---------+---------+-------+
% | 000001 | Alberto                  | 5,8     | 7,2     | 6,5   |
% | 000002 | Bárbara                  | 1,7     | 7,8     | 4,8   |
% | 000002 | Danilo                   | 6,0     | 1,8     | 3,9   |
% | 000002 | Lucas                    | 2,8     | 8,8     | 5,8   |
% | 000002 | Rejane                   | 1,1     | 5,4     | 3,2   |
% | 000002 | Talita                   | 1,3     | 6,1     | 3,7   |
% +--------+--------------------------+---------+---------+-------+

\begin{table}
  \centering
  \caption{Notas}
  \begin{tabular}{|c|l|r|r|r|}
    \hline
    RA     & Aluno                    & Nota P1 & Nota P2 & Média \\ \hline
    000001 & Alberto                  & 5,8     & 7,2     & 6,5 \\
    000002 & Bárbara                  & 1,7     & 7,8     & 4,8 \\
    000002 & Danilo                   & 6,0     & 1,8     & 3,9 \\
    000002 & Lucas                    & 2,8     & 8,8     & 5,8 \\
    000002 & Rejane                   & 1,1     & 5,4     & 3,2 \\
    000002 & Talita                   & 1,3     & 6,1     & 3,7 \\ \hline
  \end{tabular}
\end{table}
\end{document}
